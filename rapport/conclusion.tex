\chapter{Conclusion}

Les predictions sont malheureusement loin d'être parfaites, nous pensons que cela est du au manque de données. En effet, nous avons en effet effectué la collecte des données sur dix jours. Les données ne contiennent pas d'information sur l'état du réseau, ce qui pourrait aider l'algorithme de machine learning. Nous pouvons cependant retenir différentes informations grâce aux graphiques que nous avons générés. Premièrement, on remarque qu'il y a réellement très peu de retard, que le délais après 19 heure s'il y a un retard, il est toujours soit de -10 minutes ou de 10 minutes. On voit également qu'il y a une augmentation des retards aux heures de pointe, il y a également une augmentation à 21 heure à cause du confinement. On remarque que la visibilité est linéairement corrélée avec le retard. Sur le graphique disponible en annexe \ref{appendix:delay_day}, on voit bien qu'il y a moins de retard le dimanche mais qu'il y plus de problèmes le samedi. On remarque également sur le graphique des retards par stop (voir annexe \ref{appendix:delay_stop}) que les stop n°5512 et n°6474F ont des retards plus importants. Quant à notre algorithme de machine learning, il permet de prévoir des retards mais avec une précision relativement faible, il pourrait cependant être amélioré en utilisant des algorithmes plus puissants, mais surtout en améliorant la qualité et la quantité des données. Nous avons conclu que la classification, une fois améliorée, permettrait d'obtenir des informations plus utiles et plus précises.Ainsi, Ces informations permettraient de fournir à la STIB un outil pour prédire la vulnérabilité d'un stop du réseau à l'avance, des opérations préventives pourrait donc être misent en place pour réduire le retard.
