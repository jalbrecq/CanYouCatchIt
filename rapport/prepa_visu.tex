\chapter{Préparation et visualisation des données}

\section{Récolte des données}
La première étape est de toute évidence la récolte des données. Notre projet nous demandais d'avoir accès à un historique de retard mais malheureusement cette historique ne fait pas partie des datasets des opendata STIB. Nous avons donc développé un script python\footnote{Le code de ce script est disponible à l'adresse suivante: \href{https://github.com/jalbrecq/CanYouCatchIt/blob/main/sandbox/delay_gathering/delay_gathering.py}{lien github du script}} nous permettant de constituer cet historique de retard. Pour obtenir le délais, le script compare le temps d'arrivée théorique (qui nous est fournis par les fichiers GTFS\footnote{General Transit Feed Specification}) et l'heure d'arrivée prévue (qui nous est fournie par l'api "\textit{waiting time}"). Le délais est enregistré dans un fichier csv. En plus du délais, le script enregistre la température, la vitesse du vent, l'humidité et la visibilité grâce à l'api OpenWeather\footnote{Documentation disponible \href{https://openweathermap.org/}{ici}}. Un nouveau fichier csv est généré chaque jour.

Nous avons dans un premier temps récolté les données pour deux stop (les numéros 0089 et 6608G, voir leur emplacement dans l'annexe \ref{appendix:stop_pos_1}) du premier novembre au douze novembre. Les fichiers csv sont disponibles sur le \textit{repository}\footnote{Disponible \href{https://github.com/jalbrecq/CanYouCatchIt/tree/main/sandbox/data/csv}{ici}} du projet. Dans un second temps, nous avons récolté les données de tout les stops d'une ligne de bus (la ligne 39). La position de tout les stops de la ligne sont visible sur l'annexe \ref{appendix:stop_pos_2}\footnote{Une carte GoogleMyMaps est également disponible \href{https://www.google.com/maps/d/edit?mid=1_qNGPUfuZXrqC3UZXkmDOWuhEHJfYAox&usp=sharing}{ici}}. Les données récoltées durant cette deuxième phase sont disponible sur le \textit{repository} du projet\footnote{Disponible \href{https://github.com/jalbrecq/CanYouCatchIt/tree/main/sandbox/data/csv2}{ici}}.

\section{Premier coup d'œil à la structure des données}
La commande \lstinline!data.info()! indique qu'il y a 12798 lignes sans valeur pour la colonne \lstinline!delay!. Pour chacune des \textit{features} un graphique du nombre d'occurrences par valeur a été créé (voir l'annexe \ref{appendix:plots}). On peut voir sur ces graphiques que plusieurs \textit{features} ont toujours la même valeur. On peut également voir que des retards on été enregistré pour d'autres lignes que la numéro 39, il faudra donc supprimer ces dernières.

Dans le graphique en annexe \ref{appendix:plots} on remarque qu'il y a moins d'occurrences de valeur quatre pour la \textit{features} \lstinline!hour!. Cela est du au fonctionnement de l'API de la STIB. On peut également voir qu'il y a plus de d'occurrences pour les valeurs entre dix et quinze de la \textit{features} \lstinline!hour!, cette différentes est due à l'heure de démarrage et d'arrêt du script de récolte des données.

Le boxplot en annexe \ref{appendix:boxplot} montre la répartition des retards du 19 septembre sur la ligne 39 au stop 0089. On remarque que la majorité des valeurs de délais se situe entre une minute d'avance et une minute de retard.

\section{Préparation des données}
Les lignes du dataset pour les quelles la colonne \lstinline!delay! n'avait pas de valeur ont été supprimée. Les lignes dont la valeur de la colonne \lstinline!line! n'était pas égale à 39 ont également été supprimée. Les features ayant toujours la même valeur ont été supprimées. Les colonnes \lstinline!trip!, \lstinline!theoretical_time!, \lstinline!expectedArrivalTime! et \lstinline!date! ont été supprimées car elles sont pas été jugée utile. Les colonnes \lstinline!theoretical_time!, \lstinline!expectedArrivalTime! et \lstinline!date! ont été supprimées car les valeurs étaient du type \lstinline!string!. La colonne \lstinline!date! a été considérée comme redondante, sa valeur étant déjà stockée dans les colonnes \lstinline!hour!, \lstinline!minute! et \lstinline!day!.

\subsection{Stratification des données}
La stratification des données ne nous a pas semblé utile car le dataset n'est assez grand pour rendre cette dernière nécessaire. Cependant dans un but pédagogique nous avons quand même stratifier la colonne \lstinline!hour!, afin que la répartition des différentes valeurs reste identique dans le dataset ainsi que dans le test-set.

\section{Visualisation des données}
\subsection{Création du test-set}
Cela pourrait paraître étrange de mettre de coté une partie des données à ce moment. Les données n'ont même pas encore été vraiment visualisée et nous devons encore en apprendre plus avant de choisir quel algorithmes utiliser. Cependant si le test-set est créé maintenant c'est pour éviter le \textit{snooping bias}. Le test-set et constituer de vingt pourcent des données du dataset. 

\subsection{Premières visualisations}
La première visualisation générée (annexe \ref{appendix:delay_per_hour}) est la répartition des délais en fonction de l'heure. On remarque qu'après dix-huit heure on a soit une avance ou un retard de vingt minutes ou une variation de cinq minutes par rapport à l'horaire théorique, sans valeur intermédiaire. On remarque également que la majorité des délais ont une valeur nulle. On peut également voir qu'il une augmentation des délais après quinze heure jusqu'a dix-neuf heure.

Le second graphique (annexe \ref{appendix:mean_delay_per_hour}) indique le délais moyen par heure. On y remarque une augmentation des délais entre six et neuf heure, à quinze heure et ainsi qu'a vingt-et-une heure. Le pick de retard de vingt-et-une heure vient sans doute du couvre feu de vingt-deux heure, les autres pick quand à eux sont à priori du au traffic de Bruxelles.

Sur la dernière infographie (annexe \ref{appendix:mean_temp_per_hour}) on remarque une hausse des températures de midi à seize heure.

\subsection{Recherche de corrélation}
Étant donné que le notre dataset n'est pas trop grand nous pouvons facilement calculer le \textit{coefficient standard de corrélation} entre chaque paire de \textit{feature}. Comment on peut le voir avec le bout de code suivant (Voir listing \ref{cscDelay}), le retard est très peu linéairement corrélé avec les autres \textit{features}.

\begin{lstlisting}[language=Python, caption=Coefficient standard de corrélation pour la \textit{feature} \lstinline!delay!., label=cscDelay]
>>> corr_matrix = data.corr()
>>> corr_matrix["delay"].sort_values(ascending=False) # warning: this check only linear correlation
delay         1.000000
rain          0.018600
temp          0.009013
wind          0.002024
hour         -0.000627
minute       -0.006874
humidity     -0.016347
visibility   -0.034723
day          -0.079071
Name: delay, dtype: float64
\end{lstlisting}

La \textit{heatmap} (disponible en annexe \ref{appendix:corr_mat}) nous permet de confirmer que la \textit{feature} \lstinline!delay! n'est pas linéairement corrélée avec les autres \textit{features}. Elle nous fournis cependant des informations supplémentaires comme le fait que la \textit{feature} \lstinline!temp! est linéairement corrélée avec la \textit{feature} \lstinline!wind!, ainsi qu'inversément linéairement corrélée avec la \textit{feature} \lstinline!humidity!. Cependant ces corrélations reste faibles, comme on peut le voir sur la figure \ref{appendix:scatter_matrix}. On voit bien que globalement quand la valeur de la \textit{feature} \lstinline!humidity! chute quand celle de la \textit{feature} \lstinline!temp! augmente mais on reste cependant loin d'une belle ligne droite.

\subsection{Combinaison de \textit{features}}
La \textit{feature} \lstinline!hour_minute! est une nouvelle \textit{feature} que nous avons créé en combinant les \textit{features} \lstinline!hour! et \lstinline!minute!. Ce regroupement des \textit{features} est fait dans un bute pédagogique car de manière générale on a plutôt tendance à séparer les \textit{features}. La séparation des \textit{features} est une bonne façon de les rendre plus utilisable par l'algorithmes de machine learning car la plus part du temps les datasets possède des colonnes du type \lstinline!string! qui viole le principe de \textit{tidy data}. La division des \textit{features} permet améliorer les performances du modèle en découvrant des informations potentielles. Ce qui a déjà été fait en divisant la \textit{feature} \lstinline!date! en les \textit{features} \lstinline!year!, \lstinline!month!, \lstinline!day!, \lstinline!hour! et \lstinline!minute!.

\subsubsection{Visualisation de la nouvelle \textit{feature}}
Le graphique en annexe \ref{appendix:delay_per_hour_and_minute} nous montre la répartition des délais en fonction de l'heure de la journée. On remarque que le graphique garde évidement sa forme de fourche caractéristique.

En réaffichant les coefficient standard de corrélation en prenant en compte cette nouvelle \textit{feature}, on remarque qu'elle est encore moins linéairement corrélée avec la \textit{feature} \lstinline!delay! que ne le sont les \textit{features} \lstinline!hour! et \lstinline!minute!.

\section{Les \textit{features} de type \lstinline!string!}
La plus part des algorithmes de machine learning préfère travailler avec des nombres plutôt qu'avec du texte, c'est pourquoi nous convertissons la \textit{feature} \lstinline!stop! en différente catégorie représentée par un nombre.

\section{Pipeline}
Un pipeline a été créé pour faciliter l'exécution des étapes de transformation des données. Le pipeline effectue les opérations suivantes, premièrement amputer les lignes ayant au moins une valeur nulle dans l'une des colonnes. En suite vient l'étape d'ajout de la \textit{feature} \lstinline!hour_and_minute! qui peut être déactivée à la volée à l'aide d'un paramètre. La dernière étape est la création des catégories des stops comme vu dans la section précédente.