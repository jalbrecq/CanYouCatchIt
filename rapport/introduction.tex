\chapter{Introduction}
Le but de ce rapport est d'expliquer la démarche et la méthodologie qui a guidé l'élaboration des modèles de machine learning et de l'analyse des données fournie par les \textit{opendata stib-mivb}. Ce rapport est constitué des différentes parties: l'analyse et la préparation des données, l'entraînement des modèles de régression et de classification et l'analyse de leurs résultats. Nous nous sommes concentré uniquement sur une ligne de tram mais le système pourrait facilement être étendu au reste du réseau.

L'étape d'analyse et la préparation des données met en lumière les notions de normalisation, la détecteur d'\textit{outliers}, la selection de \textit{features}. La visualisation des données est également une partie importante de l'analyse des données. En suite dans l'étape d'entraînement des modèles passe par un phase de selection des méta-paramètres et d'optimisation des prédictions.

\section{Présentation du projet}
Il nous a été demandé de développer un nouveau service ou une analyse pertinente par rapport au défis de la mobilité. Plusieurs opendata nous était proposées, nous avons décidé de choisir celle de la STIB. Nous avons choisi de mettre en place une interphase permettant de savoir si prochain tram qui arrivera à un stop que l'on attend aura du retard ou non.