\chapter{Introduction}
Le but de ce rapport est d'expliquer la démarche et la méthodologie qui ont guidé l'élaboration des modèles de machine learning et de l'analyse des données fournies par les \textit{opendata stib-mivb}. Ce rapport est constitué de différentes parties: l'analyse et la préparation des données, l'entraînement des modèles de régression et de classification et l'analyse de leur résultat. Nous nous sommes concentrés uniquement sur une ligne de tram mais le système pourrait facilement être étendu au reste du réseau.

L'étape d'analyse et la préparation des données mettent en lumière les notions de normalisation, la détection d'\textit{outliers}, la selection de \textit{features}. La visualisation des données est également une partie importante de l'analyse des données. Ensuite l'étape d'entraînement des modèles passe par un phase de selection des méta-paramètres et d'optimisation des prédictions.

\section{Présentation du projet}
Il nous a été demandé de développer un nouveau service ou une analyse pertinente par rapport au défi de la mobilité. Plusieurs opendata nous étaient proposées, nous avons décidé de choisir celle de la STIB. Nous avons choisi de mettre en place une interphace permettant de savoir si le prochain tram qui arrivera au stop qu'on attend aura du retard ou non.