\chapter{Modèles et résultats}
Ce chapitre est divisé en deux parties: la première concerne la partie de l'équipe "regression" (constituée de Jean A. et de Cyprien L.); la seconde concerne la partie de l'équipe "classification" (constituée d'Antoine P. et de Jessica D.). Chacune des équipes est partie sur un vision différente du problème, comme leur nom l'indique, la première équipe a vu le problème comme un problème de régression tandis que la seconde comme un problème de classification.

\section[Partie 1 - Regression]{Partie 1 - Regression \footnote{Le code concernant cette section est disponible \href{https://github.com/jalbrecq/CanYouCatchIt/blob/main/machine_learning/notebook/CanYouCatchIt_3of3.ipynb}{ici}}}

Différent modèles ont été entraînés afin de sélectionner le meilleur d'entre eux. Le premier à avoir été testé est le modèle de regression linéaire, les prédictions obtenues sont disponibles dans le tableau en annexe \ref{appendix:linearReg}. Le modèle fonctionne mais n'est pas précis du tout. Pour nous rendre compte à quel point le modèle se trompe nous calculons la  \textit{root-mean-square error} (RMSE), cette dernière est égale à 2.47. C'est vraiment impressionnant, mais le modèle ne serait-il pas en,train d' \textit{overfiter} les données. Lorsqu'on utilise donc un modèle d'apprentissage par arbre de décision et que l'on calcule la RMSE on obtient une valeur nulle. Ce qui signifie que le modèle \textit{overfit} également les données d'entraînement.

\subsection{Cross-validation}
Afin de mieux évaluer les modèles testés, nous avons utilisé une \textit{10-fold cross-validation}. En utilisant la cross validation sur le modèle linéaire nous obtenons toujours une erreur moyenne de 4.8 pour les dix \textit{folds}. Le modèle d'apprentissage par arbre de décision obtient lui une valeur de 5.78, pire que le modèle de régression linéaire donc. Ce qui indique que l'arbre de décision \textit{overfit} tellement qu'il fonctionne moins bien que le modèle de régression linéaire. Après avoir entraîné un modèle de type \textit{Random Forest} et du type \textit{Support Vector Regression} sur 10 \textit{folds} nous obtenons respectivement une erreur moyenne de 4.57 et de 4.84. Pour la suite, nous avons choisi le modèle \textit{Random Forest}.

\subsection{\textit{Grid search}}
Nous avons donné un liste de valeur pour chaque métaparamètre (\lstinline!n_estimators! , \lstinline!max_features! et \lstinline!bootstrap!) à tester. \textit{Grid search} nous indique que la meilleur combinaison de valeur est \lstinline!{'max_features': 8, 'n_estimators': 30}! avec une erreur moyenne de 4.47. Ce résultat peut encore être amélioré car la valeur des métaparamètres \lstinline!n_estimators! et \lstinline!max_features! sélectionnés ont tous deux la valeur maximale que nous avions donnée. Nous changeons donc la liste des valeurs de \lstinline!n_estimators! et \lstinline!max_features! pour des valeurs plus grandes que 30 et 8 respectivement et ainsi de suite. La meilleur combinaison était \lstinline!{'max_features': 9, 'n_estimators': 120}! avec une erreur moyenne de 4.42 (de légères améliorations restaient cependant possible mais les gains en précisions étaient négligeables).

\subsection{\textit{Randomized Search}}
Nous avons testé une \textit{Randomized Search} avec des valeurs entre 1 et 200 pour \lstinline!max_features! et entre 1 et 8 \lstinline!n_estimators!. La meilleure combinaison des métaparamètres est \lstinline!{'max_features': 7, 'n_estimators': 180}! avec une erreur moyenne de 4.719.

\subsection{Intervale de confiance}
L'intervale de confiance (95\%) pour la RMSE est entre 4.42min et 4.68min
